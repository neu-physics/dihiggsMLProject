\section{Supervised Learning}
\label{sec:supervised}
Searches for specific signatures or interactions in collider data can in general be thought of as a classification problem - some known signal process must be identified and separated from some known and well-modeled set of background processes. Any iterative algorithm can then improve its ability to properly identify signal from background by comparing its predictions to the true known classifications and adjusting its internal parameters. This type of approach is known as supervised machine learning, and it is particulary relevant for measuring dihiggs decays. 

% BDT
\subsection{Boosted Decision Trees}
\label{sec:BDT}
talk about tress. Look into forests. Don't lose one for the other.


% ff Neural Network
\subsection{Feed Forward Neural Network}
\label{sec:NN}
Fully connected or feed-forward neural networks (ffNN or NN) have a long history in high energy physics. One of the earliest applications of this type of approach was in a search for top quark production using the CDF experiment at the Tevatron. The fundamental element of a feed-forward neural network is called a `layer`, and these layers connect some input variables to an predicted outcome which can be evaluated against some truth value. The hidden layers between input and output vectors are composed of a series of trainable activation functions and weights that allow the network to identify and iteratively combine important features of the input space. A vector of relevant physics-level information (e.g. mass of two highest \pt jets, angle between measured objects, etc) is constructed for each event, and these vectors are then `fed forward` through multiple layers in order to predict whether the event comes from a signal dihiggs process or a background QCD process.

The neural network framework used for dihiggs classification is composed of an input layer of twelve variables, two hidden layers with 175 and 90 nodes respectively, and finally a two-dimensional output layer. A schematic flowchart of the network structure is shown in Figure~\ref{fig:nn}. The input variables were selected by calculating the KS-score between signal and background distributions and then keeping the variables with the largest one-dimensional separation. The distributions for all variables used in training are shown in Figure~\ref{fig:inputVariables}.

\begin{figure}[!h] 
\begin{center}
\includegraphics*[width=0.75\textwidth] {ffNN/figures/flowchart_ffNN.png}
\caption{Structure of the feed-forward neural network. The input variables are fed through two fully connected dense layers to classify events. One dropout layer and one batch normalization layer help mitigate over-fitting during training.}
  \label{fig:nn}
\end{center}
\end{figure}

The hyperparameters of the feed-forward NN ($N_{nodes}$ in each hidden layer, learning rate, regularization, etc) were optimized by something more rigorous than trial and error. Describe the process semi in-depth and talk about what effect different parameters had on the overall performance.


% ff Neural Network
\subsection{Convolutional Network}
\label{sec:CNN}
images rule


\subsection{Residual Network}

\subsection{Lorentz Boost Network}
\label{sec:LBN}
a lorentz boost network? sounds cool, what's that?


% EFN network
\subsection{Energy Flow Network}
\label{sec:EFN}
Energy Flow Networks (EFN) and Particle Flow Networks (PFN) are algorithms that take basic jet constituents information as input rather than reconstructed jets and multi-jet composites, e.g. Higgs candidates. The EFN structure takes only the rapidity ${y}$ and azimuthal angle ${\phi}$ of jet constituents as input, while the PFN takes the rapidity $y$, azimuthal angle $\phi$, and transverse momentum $p_{T}$ of jet constituents as input. Using the constituents as input means no high level reconstruction is necessary when identifying events. Both the EFN and PFN are two-component networks, and their internal structures are shown in Figure \ref{fig:EFNArch}. The implementations of the EFN and PFN used for di-Higgs classification use 200 nodes for each hidden layer in network (a), 256 for latent space dimension and 300 nodes for each hidden layer in network (b). 

\begin{figure}[ht!]
\centering
\includegraphics[scale=0.5]{./EFN/EFNArch.png}
\caption{Network (a) takes jet constituents information as input and outputs latent space $\Phi$ for each jet constituents. Network (b) takes $\mathcal{O}$, which is the linear combination of $\Phi$, as input and outputs final result.}
\label{fig:EFNArch}
\end{figure}

The EFN/PFN networks were trained using four seperate categories split by number of jets and number of $b$-tags in order to test the network's dependence on higher-level jet information. Independent networks were trained using: all events, only events with $\geq$4 jets, only events with $\geq$4 jets and =2 $b$-tags, and only events with $\geq$4 jets and $\geq$4 $b$-tags. In each configuration, the number of signal and background events were adjusted to maintain an equal proportion of each population in the training sample. L2 regularization and dropout layers were added to minimize overfitting. The results obtained with the EFN are shown in Table~\ref{EFNtab}. The results of the PFN are shown in Table~\ref{PFNtab}.

\begin{table}[ht!]
\centering
  %\begin{center}
    \begin{tabular}{|l|c|c|c|} % <-- Alignments: 1st column left, 2nd middle and 3rd right, with vertical lines in between
      \hline\hline
      \multirow{2}{*}{\textbf{Category}} & \multicolumn{3}{c|}{0PU}\\
      \cline{2-4}
      & Best $S/\sqrt{B}$ & \textbf{N$_{\mathrm{Signal}}$} & \textbf{N$_{\mathrm{Background}}$} \\
      \hline
      All Events & $1.407 \pm 0.006$ & $1.89\cdot 10^4$ & $1.80\cdot 10^8$ \\
      4Jets & $1.363 \pm 0.006$ & $1.63\cdot 10^4$ & $1.43\cdot 10^8$ \\
      4Jets 2BTags & $1.343 \pm 0.006$ & $1.33\cdot 10^4$ & $9.95\cdot 10^7$ \\
      4Jets 4BTags & $0.867 \pm 0.008$ & $3468.65$ & $1.60\cdot 10^7$ \\
      \hline\hline
    \end{tabular}
    \caption{EFN results. Normalized to full HL-LHC dataset of 3000 fb$^{-1}$}
  %\end{center}
\label{EFNtab}
\end{table}

\begin{table}[ht!]
\centering
  %\begin{center}
    \begin{tabular}{|l|c|c|c|} % <-- Alignments: 1st column left, 2nd middle and 3rd right, with vertical lines in between
      \hline\hline
      \multirow{2}{*}{\textbf{Category}} & \multicolumn{3}{c|}{0PU}\\
      \cline{2-4}
      & Best $S/\sqrt{B}$ & \textbf{N$_{\mathrm{Signal}}$} & \textbf{N$_{\mathrm{Background}}$} \\
      \hline
      All Events & $1.618 \pm 0.008$ & $1.79\cdot 10^4$ & $1.21\cdot 10^8$ \\
      4Jets & $1.580 \pm 0.008$ & $1.32\cdot 10^4$ & $7.00\cdot 10^7$ \\
      4Jets 2BTags & $1.574 \pm 0.009$ & $1.32\cdot 10^4$ & $4.85\cdot 10^7$ \\
      4Jets 4BTags & $0.903 \pm 0.009$ & $3297.34$ & $1.33\cdot 10^7$ \\
      \hline\hline
    \end{tabular}
    \caption{PFN results. Normalized to full HL-LHC dataset of 3000 fb$^{-1}$}
  %\end{center}
\label{PFNtab}
\end{table}

The PFN provided a best significance of 1.618 when trained over all events without any cuts on the number of jets or $b$-tags.

