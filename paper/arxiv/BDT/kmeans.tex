\label{sec:kmeans}
Two additional transformations of the input variables were tested to try to improve the BDT performance. The first transformation was to pass the nominal kinematic inputs through a k-means clustering stage before training the BDT. K-means clustering is an unsupervised learning algorithm that finds unlabelled groupings in the phase-space defined by the input variables. The number of clusters to fit is a user-defined hyperparameter, and three different clusterings (15, 20, 40) were tested. The second transformation involved performing a principal component analysis (PCA) decomposition on the nominal kinematic inputs before passing through the clustering step and finally the BDT. PCA is a technique for finding an orthogonal basis of input data that minimizes the variance along each new axis. No transformation was found to improve the performance of the nominal configuration, and the results are shown in Table~\ref{tab:bdtPCACluster}.

\begin{table}[h!]
\label{tab:bdtPCACluster}
\begin{center}
  %\hskip-4.0cm
    \begin{tabular}{|l|c|c|c|} % <-- Alignments: 1st column left, 2nd middle and 3rd right, with vertical lines in between
      \hline\hline
      \textbf{Method} & $S/\sqrt{B}$ & N$_{sig}$ & N$_{bkg}$ \\
      \hline
      Nominal BDT & 1.84 $\pm$ 0.09 & 986.3  & 2.9$\cdot 10^5$ \\
      15 Clusters & 1.29 $\pm$ 0.02 & 2100.2 & 2.7$\cdot 10^6$ \\
      15 Clusters + PCA & 1.25 $\pm$ 0.02 & 2189.5 & 3.1$\cdot 10^6$ \\         
      20 Clusters & 1.30 $\pm$ 0.02 & 2260.6 & 3.0$\cdot 10^6$ \\
      20 Clusters + PCA & 1.27 $\pm$ 0.03 & 21756.4 & 1.9$\cdot 10^6$ \\         
      40 Clusters & 1.44 $\pm$ 0.03 & 1704.6 & 1.4$\cdot 10^6$ \\
      40 Clusters + PCA & 1.34 $\pm$ 0.02 & 2144.5 & 2.0$\cdot 10^6$ \\         
      \hline\hline
    \end{tabular}
    \caption{Significance and yields showing BDT performance for the nominal kinematic inputs, clustered kinematic inputs, and clustered inputs from a PCA decomposition. All yields are normalized to full HL-LHC dataset of 3000 fb$^{-1}$.}
    \end{center}
\end{table}
