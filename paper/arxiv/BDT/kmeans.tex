\label{sec:kmeans}
K-means clustering is an unsupervised learning algorithm that finds natural unlabeled groupings in the phase-space of the inputs. The k-means approach creates clusters by defining cluster centroids and associating events to the closest nearby centroid. The centroid positions are iteratively improved by minimizing the ensemble distance of all events to their associated clusters. Combining unsupervised clustering with the supervised structure of a BDT converts the unsupervised approach into a semi-supervised algorithm whose performance can be compared to other supervised methods.

The number of clusters to fit is a user-defined hyperparameter, and three different clusterings (15, 20, 40) were tested. Two unsupervised transformations of the input variables were tested to try to improve the performance of the nominal BDT. The first approach was to pass all reconstructed kinematic inputs through a k-means clustering stage before training the BDT. The second transformation involved performing a principal component analysis (PCA) decomposition on the nominal kinematic inputs before passing through the clustering step and the BDT. PCA is a technique for finding an orthogonal basis of the input data that minimizes the variance along each new axis. No transformation was found to improve the performance of the nominal configuration, and the results are shown in Table~\ref{tab:bdtPCACluster}.

\begin{table}[h!]
\label{tab:bdtPCACluster}
\begin{center}
  %\hskip-4.0cm
    \begin{tabular}{|l|c|c|c|} % <-- Alignments: 1st column left, 2nd middle and 3rd right, with vertical lines in between
      \hline\hline
      \textbf{Method} & $S/\sqrt{B}$ & N$_{\textrm{sig}}$ & N$_{\textrm{bkg}}$ \\
      \hline
      Nominal BDT & 1.84 $\pm$ 0.09 & 986.3  & 2.9$\cdot 10^5$ \\
      15 Clusters & 1.29 $\pm$ 0.02 & 2100.2 & 2.7$\cdot 10^6$ \\
      15 Clusters + PCA & 1.25 $\pm$ 0.02 & 2189.5 & 3.1$\cdot 10^6$ \\         
      20 Clusters & 1.30 $\pm$ 0.02 & 2260.6 & 3.0$\cdot 10^6$ \\
      20 Clusters + PCA & 1.27 $\pm$ 0.03 & 21756.4 & 1.9$\cdot 10^6$ \\         
      40 Clusters & 1.44 $\pm$ 0.03 & 1704.6 & 1.4$\cdot 10^6$ \\
      40 Clusters + PCA & 1.34 $\pm$ 0.02 & 2144.5 & 2.0$\cdot 10^6$ \\         
      \hline\hline
    \end{tabular}
    \caption{Significance and yields showing BDT performance when using the nominal kinematic inputs, clustered kinematic inputs, and clustered inputs from a PCA decomposition. All yields are normalized to full HL-LHC dataset of 3000 fb$^{-1}$.}
    \end{center}
\end{table}
