%Dihiggs physics is cool. Also complicated but not that complicated. Mostly just rare. Show some diagrams. Talk about rates.
\section{Dihiggs Physics}
\label{sec:physics}
The Higgs boson is an essential part of the SM and is a product of the mechanism responsible for electroweak symmetry breaking. Along with the interaction of the Higgs with the other particles of the Standard Model, the SM predicts the interaction of the Higgs boson with itself at tree-level (self-interaction). This mechanism contributes to non-resonant Higgs boson pair production together with quark-loop contributions via Yukawa-type interactions. Figure~\ref{fig:nr_hh_production} shows a schematic diagram of non-resonant Higgs boson pair production. Since the production cross section for Higgs boson pair production is extremely small within the Standard Model, 
\begin{equation*}
\sigma_{hh}\text{ (13 TeV)} = 33 \text{ fb},
\end{equation*}
any significant enhancement would indicate the presence of new physics.

\begin{figure}[!h] 
\begin{center}
\includegraphics*[width=0.75\textwidth] {dihiggsPhys/figures/nr-diHiggs-production.png}
\caption{Leading order Feynman diagrams for non-resonant production of Higgs
  boson pairs in the Standard Model through (a) the Higgs boson self-coupling
  and (b) the Higgs-fermion Yukawa interaction.} 
  \label{fig:nr_hh_production}
\end{center}
\end{figure}

Many extensions of the SM predict the existence of additional scalar bosons which may have mass larger than twice the Higgs mass and can decay into a Higgs boson pair. Searching for resonances in the $hh$ mass spectrum can help us discover or limit exotic models which predict the presence of such particles. More importantly, measuring the SM dihiggs cross-section (or placing limits on its magnitude) allow us to probe the self-coulpling of the Higgs field and better understand the mechanism behind electroweak symmetry breaking.

The following work is focused on techniques for distinguishing non-resonant (SM-like) Higgs boson pair production where both Higgs bosons decay via $h \to b \bar{b}$. The choice of using the $4b$ decay mode provides the largest possible amount of signal events but requires powerful background reduction techniques due to the large production cross-section of fully hadronic QCD processes. All results are quoted for simulated events produced by $pp$ collisions with a center-of-mass energy of 14 TeV and scaled to the full design luminosity of the HL-LHC (an integrated luminosity of 3000 fb$^{-1}$). Simulated samples are produced with Madgraph~\cite{Alwall:2014hca} and recontructed in Delphes~\cite{de_Favereau_2014} using an approximation of the upgraded Phase-II CMS detector.

\subsection{Event Reconstruction}
We tried a lot of different methods to build dijet pairs for $hh$ reconstruction. This inherently involves making some choices, and no method is 100\% efficient. Sometimes one or another is better for a given method. What's cool is that some methods don't require reconstrucion at all.
