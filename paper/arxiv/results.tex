\section{Results}
\label{sec:results}

The methods covered in this paper are by no means an exhaustive review of the ML landscape available to high energy physics. Still, a wide range of techniques and philosophies are covered. Table~\ref{tab:summary} provides a summary of the methods described in the previous sections. The results of a traditional 1-D sequential cut technique is shown for comparison though the details have not been discussed in the previous sections. Clear gains in sensitivity compared to this baseline are apparent for many of the ML models tested in this review.

\begin{table}[h!]
\label{tab:summary}
  \begin{center}
  \begin{tabular}{|l|c|c|c|} % <-- Alignments: 1st column left, 2nd middle and 3rd right, with vertical lines in between
      \hline\hline
      \multirow{2}{*}{\textbf{Method}} & \multicolumn{3}{c|}{0PU} \\
      \cline{2-4}
      & Best $S/\sqrt{B}$ & \textbf{N$_{\mathrm{Signal}}$} & \textbf{N$_{\mathrm{Background}}$} \\
      \hline
      Autoencoder           & 0.81 $\pm$ 0.01 & 5840.8 & 5.2 $\cdot$ $10^7$ \\
      1D-Rectangular Cuts   & 0.82 $\pm$ 0.02 & 3621.0 & 1.97 $\cdot$ $10^7$ \\
      k-Means Clustering    & 1.44 $\pm$ 0.02 & 1703.6 & 1.4 $\cdot$ $10^6$ \\
      Particle Flow Network & 1.62 $\pm$ 0.01 & 1.8 $\cdot$ $10^4$ & 1.2 $\cdot$ $10^8$ \\
      Boosted Decision Tree & 1.84 $\pm$ 0.09 & 986.3  & 2.8 $\cdot$ $10^5$ \\
      %Lorentz Boost Network & 1.87 $\pm$ 0.08 & 1123.3 & 3.6 $\cdot$ $10^5$ \\
      Feed-Forward NN       & 2.40 $\pm$ 0.08 & 1659.9 & 4.8 $\cdot$ $10^5$ \\
      Random Forest         & 2.44 $\pm$ 0.19 & 544.7 & 5.0 $\cdot$ $10^4$ \\
      Convolutional NN      & 2.85 $\pm$ 0.02 & 1.0 $\cdot$ $10^4$ & 1.3 $\cdot$ $10^7$ \\
      \hline\hline
    \end{tabular}
    \caption{Comparison of method significance and signal/background yields normalized to full HL-LHC dataset of 3000 fb$^{-1}$.}
  \end{center}
\end{table}

An important caveat to keep in mind is that all results discussed here were determined in conditions with zero pileup. In higher pileup environments like those expected at the HL-LHC, reconstruction algorithms see serious reductions in correct combinatoric matching. This effect will certainly degrade the expected performance of techniques that rely on explicit event reconstruction. Methods that do not rely on event reconstruction (CNN, PFN) might be more robust to these effects, and this should be studied in further work. The unsupervised AE technique performed the worst among all the methods tested, but this is likely a reflection of the fact that model-specific methods often outperform model-unspecific methods when evaluated on the model used in training.

