\section{Introduction}
\label{sec:intro}

The use of machine learning (ML) techniques in high energy particle physics has rapidly expanded in the last few decades. The proliferation of methods and applications has touched nearly every segment of analysis and reconstruction~\cite{albertsson2018machine} and will be vital in understanding the full dataset of the Large Hadron Collider (LHC) and data from future colliders.

Common approaches involve using linear techniques like decision trees and non-linear approaches like neural networks. These techniques are then used to reconstruct objects like leptons and jets, to tag b-quarks and boosted decays, and to classify different processes. Many models depend on kinematic input features physicists have traditionally used; other architectures rely on emergent features produced in a more abstract phase-space. With so many usable machine-learning options available, interesting questions arise about what types of information are best to feed to our networks. The information that is best for physicists to learn from might not be optimal for sophisticated computing algorithms.
%You can even think of fun semi-supervised approaches like clustering and maybe some stuff with autoencoders. To be totally honest, I'm not really sure if the autoencoder stuff is relevant for this paper, but it's really interesting and is maybe cool for unexpected BSM signatures? Could be a cool side-study.

The goal of this paper is to explore a wide range of current ML techniques that can be used to identify di-Higgs production at the High Luminosity LHC (HL-LHC). Observing di-Higgs production is necessary to measure the self-coupling of the Higgs boson and fully understand the nature of electroweak symmetry breaking. The difficulty in measuring Higgs pair production lies in the tiny cross-section of even the largest branching fraction ($hh\rightarrow b\bar{b}b\bar{b}$) and the relative abundance of similarly reconstructed QCD events.

This paper is organized as follows: Section 2 deals with the physics relevant for di-Higgs production and the QCD background. Sections 3 and 4 summarize the various ML methods tested. The maximum significance
\begin{equation}
  \sigma = \frac{N_{\textrm{signal}}}{\sqrt{N_{\textrm{background}}}}
\end{equation}
is provided for each method along with event yields normalized to the HL-LHC design integrated luminosity of 3000 fb$^{-1}$. Section 5 compares the results from the various approaches.
