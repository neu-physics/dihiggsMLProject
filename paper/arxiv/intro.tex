\section{Introduction}
\label{sec:intro}

The use of machine learning (ML) techniques in high energy particle physics has rapidly increased since its first use at the Tevatron~\cite{Abazov_2009}. The proliferation of techniques and applications has touched nearly every segment of analysis and reconstruction~\cite{albertsson2018machine} and will be vital to understand the full dataset of the LHC and data from future colliders.

Common approaches involve using linear techniques like decision trees as well as non-linear approaches like neural networks. These techniques are then used to reconstruct objects like leptons and jets, to tag objects like b-quarks or boosted decays, and classify different processes. Many models are built with the same kinematic input features physicists typically use while other architectures rely on emergent features produced in a more abstract phase-space. Regardless of the approach, the rise of so many approaches raises interesting questions about what types of information are best to feed to our networks-- things that are best for physicists to learn from might not be optimal for sophisticated computing algorithms.
%You can even think of fun semi-supervised approaches like clustering and maybe some stuff with autoencoders. To be totally honest, I'm not really sure if the autoencoder stuff is relevant for this paper, but it's really interesting and is maybe cool for unexpected BSM signatures? Could be a cool side-study.

The goal of this paper is to explore a wide range of current ML techniques for applicability at identifying dihiggs production at the HL-LHC. Observing dihiggs production is necessary to measure the self-coupling of the Higgs boson and fully understand the nature of electroweak symmetry breaking. The difficulty in measuring dihiggs production lies in the tiny cross-section of even the largest branching fraciton ($hh\rightarrow b\bar{b}b\bar{b}$) and the relavtive abundance of similarly reconstructed QCD events. Section 2 deals with the physics relevant for dihiggs production and the QCD background. Sections 3 and 4 summarize the various ML methods tested, and Section 5 compares the results from the various approaches.
